\chapter{Introduction} % (fold)
\label{cha:introduction}


\section{Historique} % (fold)
\label{sec:historique}

Chomsky est un linguiste américain voulant formaliser le langage naturel. Grâce aux grammaires génératrices et transformationnelles, en 1950, il est à réussi à créer la hiérarchie de Chomsky, permettant de classer les langages.

En 1960, Schützenber et Mivat ont développé les langages informatiques grâce aux travaux de Chomsky.

% section historique (end)


\section{Utilité} % (fold)
\label{sec:utilit_}

L'utilisation des travaux de Chomsky pour les langages informatiques servent comme :
\begin{itemize}
	\item D'Outil théorique, permettant d'écrire des règles d'un langage, grâce aux grammaires.
	\item L'analyse syntaxe, permettant de reconnaître, si un texte respecte la syntaxe d'un langage.
	\item L'analyse sémantique, donnant un sens à un texte syntaxiquement correct.
	\item La compilation, traduisant le texte d'un langage dans un autre.
\end{itemize}

% section utilit_ (end)


% chapter introduction (end)
